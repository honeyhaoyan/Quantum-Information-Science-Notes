%!TEX program = xelatex
\documentclass[a4paper, 11pt]{article} % Font size (can be 10pt, 11pt or 12pt) and paper size (remove a4paper for US letter paper)

%\documentclass{article}
%\usepackage[protrusion=true,expansion=true]{microtype} % Better typography
\usepackage{graphicx} % Required for including pictures
\usepackage{wrapfig} % Allows in-line images
\usepackage{ctex}

\usepackage{mathpazo} % Use the Palatino font
\usepackage[T1]{fontenc} % Required for accented characters
\usepackage{fontspec}
\usepackage{xunicode}
\usepackage{xltxtra} 
\usepackage{amsmath}
\usepackage{geometry}
\usepackage[colorlinks,linkcolor=black]{hyperref}
\geometry{a4paper,scale=0.8}
\linespread{1.05} % Change line spacing here, Palatino benefits from a slight increase by default
%\linespread{0.5}
\makeatletter
\renewcommand\@biblabel[1]{\textbf{#1.}} % Change the square brackets for each bibliography item from '[1]' to '1.'
\renewcommand{\@listI}{\itemsep=0pt} % Reduce the space between items in the itemize and enumerate environments and the bibliography

\renewcommand{\maketitle}{ % Customize the title - do not edit title and author name here, see the TITLE block below

\begin{flushright} % Right align
{\LARGE\@title} % Increase the font size of the title

\vspace{50pt} % Some vertical space between the title and author name

{\large\@author} % Author name
\\\@date % Date

\vspace{10pt} % Some vertical space between the author block and abstract
\end{flushright}
}

%----------------------------------------------------------------------------------------
%	TITLE
%----------------------------------------------------------------------------------------

\title{\textbf{量子信息学}\\ % Title
lec 9} % Subtitle

\author{\textsc{郝琰 516021910721} % Author
\\{\textit{ACM Class,2016}}} % Institution

\date{\today} % Date
\begin{document}
\maketitle
\section{Class 1}
\subsection{idea 1}
$$
f: \lbrace0,1\rbrace^n \rightarrow \lbrace 0,1 \rbrace \quad f(x_0) = 1\quad f(x) = 0(x!=x_0)
$$
若用量子进行计算(搜素unique answer $x_0$),则复杂度约为$O(\sqrt{N})$(1996,Grover,PRL)\\
$O_f$的作用
$$
|x>\quad \rightarrow \quad (-1)^{f(x)}|x>
$$
\begin{itemize}
	\item input 
	$$
	|0>
	$$
	\item the first step
	$$
	H^{\bigotimes n} \quad \rightarrow \quad \sum^{2^n-1}_{x=0} \frac{|x>}{\sqrt{2^n}}
	$$
	\item the second step
	$$
	O_f
	$$
	\item the third step
	$$
	U_f
	$$
	Jump to the second step
\end{itemize}	

The point is , how to deciede what $U_f$ is. $U_f$的作用在于下次更方便地查询
$$
U = 2|\psi><\psi| - I = H^{\bigotimes n}(2|0^{\bigotimes n}><0^{\bigotimes n}|H^{\bigotimes n})
$$
把O和U和起来作为G
$$
G^k|\psi> \approx |x_0>
$$
因此复杂度约为$\sqrt{N}$
$$
N = 2^n
$$

\begin{align*}
|\psi> &=\frac{1}{\sqrt{N}} \sum_{x=0}^{N-1}|x> \\
&= \frac{1}{\sqrt{N}} (\sum_{\mbox{x is not solution}}|x^{'}>+\sum_{\mbox{x is solution}}|x^{''}>)\\
& = \frac{1}{\sqrt{N}}(\sqrt{N-M}|\alpha> + \sqrt{M}|\beta>)
\end{align*}

$$
|\psi> = \sqrt{\frac{N-M}{N}}|\psi> + \sqrt{\frac{M}{N}}|\beta>
$$
$$
G = U*O
$$
$$
|\alpha > \rightarrow |\alpha> \quad |\beta> \rightarrow -|\beta>
$$
$$
O = |\alpha><\alpha| - |\beta><\beta| = 2|\alpha><\alpha| - I
$$
$$
U = 2|\psi><\psi| - I
$$
G即为两个反射变换的乘积,即为旋转
$$
|\psi> = cos\theta|\psi> + sin\theta|\beta>
$$
若$\psi$与X轴正向原本夹角$\theta$,则作用结束后与X轴正向夹角$3\theta$
$$
O|\psi> = cos(-\theta)|\alpha> + sin(-\theta)|\beta>
$$
$$
G|\psi> = UO|\psi> = cos(3\theta)|\alpha> + sin(3\theta)|\beta>
$$
$$
G|\psi>^k = UO|\psi> = cos((2k+1)\theta)|\alpha> + sin((2k+1)\theta)|\beta>
$$
即每次作用可看作该向量顺时针旋转$2\theta$
$$
cos(2k+1)\theta |\alpha> + sin(2k+1)\theta|\beta> \approx |\beta>
$$
$$
(2k+1)\sqrt{\frac{M}{N}} = \frac{\pi}{2}
$$
$$
k = \frac{\pi}{4}\sqrt{\frac{N}{M}} - \frac{1}{2}
$$

\section{Class 3}
\subsection{Problem 1}
面对一个黑盒$O_f \in \lbrace O_{f_1},O_{f_2},...,O_{f_N} \rbrace$
$$
O_{f_k}|x> = |x> \quad x\ne k
$$
$$
O_{f_k}|k> = |k> \quad x = k
$$
如果这些向量已经两两正交,由于不是标准正交,就做一个unitary操作,变化到标准正交基上。即变化到$\lbrace |0>,|1>,...,|d> \rbrace$\\
可以把原本的正交基作为输入X
\subsection{Problem 2}
区分
$$
U \in \lbrace I,X,Y,Z \rbrace
$$
$$
(U^A\bigotimes I^B)|\beta_{00}> = \lbrace |\beta_{00}>,|\beta_{01}>,|\beta_{10}>,|\beta_{11}> \rbrace
$$























\end{document}