%!TEX program = xelatex
\documentclass[a4paper, 11pt]{article} % Font size (can be 10pt, 11pt or 12pt) and paper size (remove a4paper for US letter paper)

%\documentclass{article}
%\usepackage[protrusion=true,expansion=true]{microtype} % Better typography
\usepackage{graphicx} % Required for including pictures
\usepackage{wrapfig} % Allows in-line images
\usepackage{ctex}

\usepackage{mathpazo} % Use the Palatino font
\usepackage[T1]{fontenc} % Required for accented characters
\usepackage{fontspec}
\usepackage{xunicode}
\usepackage{xltxtra} 
\usepackage{amsmath}
\usepackage{geometry}
\usepackage[colorlinks,linkcolor=black]{hyperref}
\geometry{a4paper,scale=0.7}
\linespread{1.05} % Change line spacing here, Palatino benefits from a slight increase by default
%\linespread{0.5}
\makeatletter
\renewcommand\@biblabel[1]{\textbf{#1.}} % Change the square brackets for each bibliography item from '[1]' to '1.'
\renewcommand{\@listI}{\itemsep=0pt} % Reduce the space between items in the itemize and enumerate environments and the bibliography

\renewcommand{\maketitle}{ % Customize the title - do not edit title and author name here, see the TITLE block below

\begin{flushright} % Right align
{\LARGE\@title} % Increase the font size of the title

\vspace{50pt} % Some vertical space between the title and author name

{\large\@author} % Author name
\\\@date % Date

\vspace{10pt} % Some vertical space between the author block and abstract
\end{flushright}
}

%----------------------------------------------------------------------------------------
%	TITLE
%----------------------------------------------------------------------------------------

\title{\textbf{Introduction to Data Science Homework}\\ % Title
hw 1} % Subtitle

\author{\textsc{郝琰 516021910721} % Author
\\{\textit{ACM Class,2016}}} % Institution

\date{\today} % Date

%----------------------------------------------------------------------------------------

\begin{document}

\maketitle % Print the title section

\section*{Class 1}

\subsection{标准正交积(ONS)}
$$
||\psi|| = (<\psi|\psi>)^{\frac{1}{2}}
$$
$$
H = span  |e_1>, |e_2> , ... , |e_d>
$$
$$
H = span  |1>, |2> , ... , |d>
$$
$$
|\psi> = \sum^d_{i = 1}\alpha _i |i>
$$
$$
\alpha_i = <i | \psi>
$$
$$
\sum^d_{i = 1} |i><i| = I_d
$$

字空间上的单位矩阵:字空间上的投影算子\\

A : H -> H linear operator\\

The Heimition conjunction of A is defined as A daggger

$$
(|\psi >, A |\psi >) = 
$$

A : H -> $H^{'}$

$$
A = I_HAI_{H^{'}}
$$
$$
\sum_{ij}<i|A|j> |i><j|
$$
$$
\sum_{ij}<e_i|A|e_j> |e_i><e_j|
$$

$$
A = \sum_{ij}<i|A|j> |i><j|
$$
$A^{+}$ 共轭转置

$$
A^{+} = \sum_{ij}<i|A^{+}|j> |i><j|
$$
$$
A^{+} = \sum_{ij}(<i|A|j>)^{*} |i><j|
$$

$$
A = |\psi><\psi|,  
$$
\begin{itemize}
\item Hermitian operator A: H->H
\item 123
\item A is normal is $AA^{+} = A^{+}A$
 $$
A = \sum^d_{i = 1} \alpha |i><i|
$$
$$
A^2 = \sum^d_{i =1} \alpha^2 |i><i|
$$
$$
A = \sum^k_{i=1} \lambda_iP_i
$$
$$
P_iP_j = \delta_{ij}P_i, P_i^{+} = P_i , \sum P_i = I
$$
\end{itemize}

\subsubsection*{Unitary operator U}
$$
U^{+}U = I , U^{-1} = U^{+}
$$
酉变换,相当于旋转,保持内积不变
$$
(U|\psi>,U|\psi>) = (|\psi>,|\psi>)
$$
$$
U = \sum^d_{j = 1}e^{i\theta j} |j><j|
$$

\section*{Class 2}
Polar Decomposition
$$
A = JU 
$$
$$
A^{+} = (JU)^{+} = U^{+}J^{+} 
$$
$$
AA^{+} = J^2, J = (AA^{+})^{\frac{1}{2}}
$$
$$
A = UDU^{+} = ?
$$
奇异值分解 ??? 
$$
A = JU = V
$$

把A写作普分解的好处
$$
A = \sum_{i = 1}^k \lambda_iP_i
$$
$$
A^n = \sum_{i = 1}^k \lambda_i^nP_i
$$
\subsubsection*{commutator}
$$
[A,B] = AB - BA
$$
$$
[A,B]  =0, AB = BA
$$
iff
$$
A = \sum_i \lambda_i|i><i|
$$
$$
B = \sum_i \mu_i |i><i|
$$
特殊的情形,例如,B 测得 $\mu_i$ ,A 测得$\lambda_i$。这就是对译子的特征.\\

反对译子:
$$
\lbrace A,B \rbrace = AB+BA
$$
$$
\lbrace A,B \rbrace = 0, AB = -BA
$$

\section*{some 应用}
\subsection*{trace}
$$
tr(A) = \sum_{i = 1}^d <i|A|i>
$$
$$
tr(AB) = tr(BA)
$$
$$
tr(A|\psi><\psi|) = <\psi|A|\psi>
$$
???
$$
e^{i\theta \overrightarrow{n} \overrightarrow{\delta}} ???
$$

\subsection{如何将两个空间张成一个空间}
$$
H = span \lbrace |i> , i = 1,2,3,...d \rbrace
$$
$$
H^{'} = span \lbrace |i> , i = 1,2,3,...d^{'} \rbrace
$$
$$
dim(H \oplus H^{'}) = dd^{'}
$$

\clearpage
\section*{2018.7.15}
\section*{Class 1}


%A : H $\rightarrow$ H. $A^*$ is the unique linear operator such that
\subsection*{$A^+$ and some review}
\begin{itemize}
\item
A : H $\rightarrow$ H. $A^*$ is the unique linear operator such that
$$
(|\psi>, A|\psi >) = (A^+|\psi>, |\psi >)
$$
\item
construct
\begin{align*}
<i|A^+|i> & = (<i|A|i>)^* \\
A^+ &= \sum_{ij} (<j|A|i>)^*|i><j|
\end{align*}
\item
unique
$$
(|\psi>, A|\psi >) = (A^+|\psi>, |\psi >) = (B|\psi>, |\psi >)
$$
$$
(|\psi>, A^+|\psi >) = (|\psi>, B|\psi >)
$$
$$
(|\psi>, (A^+ - B)|\psi >) = 0
$$
$$
A^+ = B
$$
\end{itemize}

$A^+$:算子函数,相当于把一个算子映射为另一个算子

一些性质(和转置性质相似)

\begin{itemize}
	\item
	$$
	(A \bigotimes B)^+ = A^+ \bigotimes B^+
	$$
	\item
	$$
	(AB)^+ = B^+A^+
	$$
	\item
	tr()
	\item 
	det()
\end{itemize}

\subsection{补充}		

\subsubsection{tr()}
$$
tr(A) = \sum_{i = 1}^d <i|A|i>
$$
\begin{itemize}
	\item
	linear
	\item
	cydic
	\item f: L(H) -> C ; linear and cydic
	$$
	f = \lambda tr()
	$$
\end{itemize}

\subsubsection{全体线性算子构成一个希尔伯特空间}
$$
L(H) = \lbrace A: A is linear operator over H \rbrace
$$
$$
<A,B> = tr(A^+B)
$$

\subsubsection{Schmidt procedure}
对任意一组(n个)线性独立的向量,存在同样个数的线性正交基使得其与前者张成同样的线性空间


\subsection{idea 1: 柯西洗袜子}
$$
|<v|w>| \leq 1 \quad <v|v> = 1 \quad <w|w> = 1
$$
首先用w扩充一组标准正交基,w是这组基的第一个元素。
\begin{align*}
1 = <v|v> & = <v| \sum_{i = 1}^d |i><i| |v>\\
& = <v|w><w|v> + <v| \sum_{i = 2}^d |i><i| |v>\\
& \geq <v|w><w|v>
\end{align*}
































\end{document}