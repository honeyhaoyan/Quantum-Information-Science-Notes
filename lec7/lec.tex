%!TEX program = xelatex
\documentclass[a4paper, 11pt]{article} % Font size (can be 10pt, 11pt or 12pt) and paper size (remove a4paper for US letter paper)

%\documentclass{article}
%\usepackage[protrusion=true,expansion=true]{microtype} % Better typography
\usepackage{graphicx} % Required for including pictures
\usepackage{wrapfig} % Allows in-line images
\usepackage{ctex}

\usepackage{mathpazo} % Use the Palatino font
\usepackage[T1]{fontenc} % Required for accented characters
\usepackage{fontspec}
\usepackage{xunicode}
\usepackage{xltxtra} 
\usepackage{amsmath}
\usepackage{geometry}
\usepackage[colorlinks,linkcolor=black]{hyperref}
\geometry{a4paper,scale=0.8}
\linespread{1.05} % Change line spacing here, Palatino benefits from a slight increase by default
%\linespread{0.5}
\makeatletter
\renewcommand\@biblabel[1]{\textbf{#1.}} % Change the square brackets for each bibliography item from '[1]' to '1.'
\renewcommand{\@listI}{\itemsep=0pt} % Reduce the space between items in the itemize and enumerate environments and the bibliography

\renewcommand{\maketitle}{ % Customize the title - do not edit title and author name here, see the TITLE block below

\begin{flushright} % Right align
{\LARGE\@title} % Increase the font size of the title

\vspace{50pt} % Some vertical space between the title and author name

{\large\@author} % Author name
\\\@date % Date

\vspace{10pt} % Some vertical space between the author block and abstract
\end{flushright}
}

%----------------------------------------------------------------------------------------
%	TITLE
%----------------------------------------------------------------------------------------

\title{\textbf{量子信息学}\\ % Title
lec 7} % Subtitle

\author{\textsc{郝琰 516021910721} % Author
\\{\textit{ACM Class,2016}}} % Institution

\date{\today} % Date
\begin{document}
\maketitle
\section{Class 1}
\subsection{Preview 1}
$$
|c,t> \rightarrow |c,t\bigoplus c>
$$
若c=0,则对t不做操作,若c=1,则对t逻辑取反\\
CNOT + 单个量子比特门可以做一切操作
\section{Class 2 , 3: Simulation}
$$
|\psi(t)> = e^{-iHt}|\psi(0)>
$$
$$
|\psi(t)> = U(t)|\psi(0)>
$$
$$
U = e^{-iK}
$$
\begin{itemize}
	\item
	input
	$$
	H\quad |\psi (0)> \quad t_f \quad \epsilon
	$$
	\item
	output
	$$
	|\tilde{\psi}(t_f)> s.t.
	$$
	$$
	|<\tilde{\psi}(t_f)|e^{iHt_f}|\psi(0)>| \geq 1-\epsilon
	$$
\end{itemize}

H: 哈密顿量
$$
H = \sum_{k=1}^{poly(n)} H_k \quad \mbox{$H_k$ 是作用于常数个系统的哈密顿量}
$$
$$
H = \sum_{k = 1}^{n-1} X_k \bigoplus X_{k+1}
$$
$$
e^{i(H_1+H_2)} = e^{iH_1}*e^{iH_2} \quad if \quad [H_1,H_2] = 0
$$
$$
e^{i(H_1H_2)} = lim_{n \rightarrow \infty}(e^{i\frac{H_1}{n}}e^{i\frac{H_2}{n}})^n
$$














\end{document}