%!TEX program = xelatex
\documentclass[a4paper, 11pt]{article} % Font size (can be 10pt, 11pt or 12pt) and paper size (remove a4paper for US letter paper)

%\documentclass{article}
%\usepackage[protrusion=true,expansion=true]{microtype} % Better typography
\usepackage{graphicx} % Required for including pictures
\usepackage{wrapfig} % Allows in-line images
\usepackage{ctex}

\usepackage{mathpazo} % Use the Palatino font
\usepackage[T1]{fontenc} % Required for accented characters
\usepackage{fontspec}
\usepackage{xunicode}
\usepackage{xltxtra} 
\usepackage{amsmath}
\usepackage{geometry}
\usepackage[colorlinks,linkcolor=black]{hyperref}
\geometry{a4paper,scale=0.8}
\linespread{1.05} % Change line spacing here, Palatino benefits from a slight increase by default
%\linespread{0.5}
\makeatletter
\renewcommand\@biblabel[1]{\textbf{#1.}} % Change the square brackets for each bibliography item from '[1]' to '1.'
\renewcommand{\@listI}{\itemsep=0pt} % Reduce the space between items in the itemize and enumerate environments and the bibliography

\renewcommand{\maketitle}{ % Customize the title - do not edit title and author name here, see the TITLE block below

\begin{flushright} % Right align
{\LARGE\@title} % Increase the font size of the title

\vspace{50pt} % Some vertical space between the title and author name

{\large\@author} % Author name
\\\@date % Date

\vspace{10pt} % Some vertical space between the author block and abstract
\end{flushright}
}

%----------------------------------------------------------------------------------------
%	TITLE
%----------------------------------------------------------------------------------------

\title{\textbf{量子信息学}\\ % Title
lec 6} % Subtitle

\author{\textsc{郝琰 516021910721} % Author
\\{\textit{ACM Class,2016}}} % Institution

\date{\today} % Date
\begin{document}
\maketitle
\section{Class 1}
\subsection{review}
$$
|\psi> = \frac{|00>+|11>}{\sqrt{2}}
$$
整体状态完全确定,个体状态却未知\\
$$
\rho^A = tr_B|\psi><\psi| = \frac{1}{1}(|0><0|+|1><1|) = \frac{I^A}{2}
$$
$$
\mbox{同理}\quad \rho^B = \frac{I^B}{2}
$$
因此,该系统虽然整体状态已知,个体却是两个状态概率相当,无法得到任何信息
\subsection{idea 1 : 逻辑门}
$$
f : \lbrace0,1\rbrace^n \rightarrow \lbrace0,1\rbrace
$$
$O_f$ 输入x和辅助位y,得到x和$y\bigoplus f(x)$
$$
O_f^2 = I
$$
\section{Class 2: logical gateway}
\begin{itemize}
	\item
	$$
 I = \left[
 \begin{matrix}
   1 & 0 \\
   0 & 1 
  \end{matrix}
  \right] 
$$
X,Y,Z
\item
H,S,T
\item
$$
R_x(\frac{\theta}{2}) = e^{-i\frac{\theta}{2}X}
$$
$$
R_y(\frac{\theta}{2}) = e^{-i\frac{\theta}{2}Y}
$$
$$
R_z(\frac{\theta}{2}) = e^{-i\frac{\theta}{2}Z}
$$
$$
A^2 = I \quad e^{iXA} = cos(x) + isinx * A
$$
$$
R_x(\frac{\theta}{2}) = cos\frac{\theta}{2}*I - isin\frac{\theta}{2}X = 
$$
\end{itemize}

$$
\left[
\begin{matrix}
e^{i(\alpha+\delta)}cos\frac{\theta}{2} & e^{}

\end{matrix}
\right]
$$
$$
U = \left[
\begin{matrix}
a & b \\
c & d 
\end{matrix}
\right]
$$

任何一个门可以通过
$$
CNOT + \lbrace S,T,H \rbrace
$$
来构造
$$
U = e^{i\alpha}AXBXC \quad ABC = I
$$

$$ U =
\left[
\begin{matrix}
a & b & c\\
d & e & f\\
h & i & j
\end{matrix}
\right] 
$$
$$ U_i =
\left[
\begin{matrix}
x & y & 0\\
z & t & 0\\
0 & 0 & 1
\end{matrix}
\right] 
$$
$$ U_1U =
\left[
\begin{matrix}
a1 & b1 & c1\\
0 & e1 & f1\\
h1 & i1 & j1
\end{matrix}
\right] 
$$
$$ U_2U =
\left[
\begin{matrix}
1 & 0 & 0\\
0 & e1 & f1\\
0 & i1 & j1
\end{matrix}
\right] = U_3
$$
$$
U = U_1^+U_2^+U_3
$$

\clearpage
\section{Deutsch - Jozsa Algorithm}
$$
f : \lbrace 0,1 \rbrace^n \rightarrow \lbrace 0,1 \rbrace
$$
$$
S_0 = \lbrace x:f(x) = 0 \rbrace
$$
$$
S_1 = \lbrace x:f(x) = 1 \rbrace
$$
\begin{itemize}
	\item constant function
	$$
	f(x) = f(x^{'})
	$$
	\item balanced function
	$$
	|S_0| = |S_1|
	$$
\end{itemize}	
要判定f是常数函数还是平衡函数,量子计算只需要查询一次
\begin{itemize}
	\item
	input : $|x>,|->$
	\item
	经过$U_f$
	$$
	|x>|-> \rightarrow (-1)^{f(x)}|x>|->
	$$
	\item
	$$
	\frac{|0>+|1>}{\sqrt{2}} 
	$$


































\end{document}